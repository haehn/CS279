\documentclass{chi-ext}
% Please be sure that you have the dependencies (i.e., additional LaTeX packages) to compile this example.
% See http://personales.upv.es/luileito/chiext/

%% EXAMPLE BEGIN -- HOW TO OVERRIDE THE DEFAULT COPYRIGHT STRIP -- (July 22, 2013 - Paul Baumann)
 \copyrightinfo{Copyright is held by the author/owner(s). \\
 {\emph{CHI'14}}, April 26--May 1, 2014, Toronto, Canada. \\
Copyright \copyright~2014 ACM ISBN/14/04...\$15.00. \\
 DOI string from ACM form confirmation}
%% EXAMPLE END -- HOW TO OVERRIDE THE DEFAULT COPYRIGHT STRIP -- (July 22, 2013 - Paul Baumann)

\title{MindMargin: An Article-Adjacent Commenting Platform}

\numberofauthors{3}
% Notice how author names are alternately typesetted to appear ordered in 2-column format;
% i.e., the first 4 autors on the first column and the other 4 auhors on the second column.
% Actually, it's up to you to strictly adhere to this author notation.
\author{
  \alignauthor{
  	\textbf{Daniel Haehn}\\
  	\affaddr{School of Engineering and Applied Sciences}\\  	
  	\affaddr{Harvard University}\\
  	\affaddr{Cambridge, MA USA}\\
  	\email{haehn@seas.harvard.edu}
  }
  \vfil
  \alignauthor{
  	\textbf{Sharon Zhou}\\
  	\affaddr{Harvard College}\\
  	\affaddr{Cambridge, MA USA}\\
  	\email{zhou12@college.harvard.edu}
  }
%  \alignauthor{
%  	\textbf{Fifth Author}\\
%  	\affaddr{AuthorCo, Inc.}\\
%  	\affaddr{123 Author Ave.}\\
%  	\affaddr{Authortown, PA 54321 USA}\\
%  	\email{author6@anotherco.com}
%  }
  \vfil
  \alignauthor{
  	\textbf{Krzysztof Z. Gajos}\\
  	\affaddr{School of Engineering and Applied Sciences}\\
  	\affaddr{Harvard University}\\
  	\affaddr{Cambridge, MA USA}\\
  	\email{kgajos@eecs.harvard.edu}
  }
%  \alignauthor{
%  	\textbf{Sixth Author}\\
%  	\affaddr{AuthorCo, Inc.}\\
%  	\affaddr{123 Author Ave.}\\
%  	\affaddr{Authortown, PA 54321 USA}\\
%  	\email{author7@anotherco.com}
%  }
}

% Paper metadata (use plain text, for PDF inclusion and later re-using, if desired)
\def\plaintitle{MindMargin: Challenging Established Opinions with an Article-Adjacent Commenting Platform}
\def\plainauthor{Daniel Haehn, Sharon Zhou, Krzysztof Gajos}
\def\plainkeywords{Comments; anchored; commenting systems; opinions.}
\def\plaingeneralterms{Comments, Opinions}

\hypersetup{
  % Your metadata go here
  pdftitle={\plaintitle},
  pdfauthor={\plainauthor},  
  pdfkeywords={\plainkeywords},
  pdfsubject={\plaingeneralterms},
  % Quick access to color overriding:
  %citecolor=black,
  %linkcolor=black,
  %menucolor=black,
  %urlcolor=black,
}

\usepackage{graphicx}   % for EPS use the graphics package instead
\usepackage{balance}    % useful for balancing the last columns
\usepackage{bibspacing} % save vertical space in references


\begin{document}

\maketitle

\begin{abstract}
Commenting systems are a popular means for facilitating conversation among readers on many websites. Reading and writing comments can increase active user engagement in information exchange, personal reflection, and lively discussion, among others. We explore how user engagement can be increased by proposing a new commenting system and interface, MindMargin. In contrast to traditional commenting systems where comments are featured below the article, MindMargin presents comments adjacent to the article. Users can post and navigate comments and replies on a horizontal infinite scroll. Comments are anchored by users to specific sections or phrases of the article. This system exposes users to a diverse and relevant array of opinions as they read.
\end{abstract}

\keywords{\plainkeywords}

\category{H.5.2.}{Information Interfaces and Presentation (e.g. HCI)}{User Interfaces}

%\terms{\plaingeneralterms}

\section{Introduction}

News pages, media sites, online shops, blogs, and social networks allow their users to provide content-related feedback with comments. For these users, online commenting holds a compelling promise: to share their views, to contribute their relevant knowledge, and to engage in lively, thoughtful discussion with each other. As political discourse online increasingly polarizes and generalizes individual political views, the prevailing opinion among researchers has been to present people with diverse perspectives in order to encourage more nuanced, independently reached views~\cite{ConsiderIt,Politics,NewsCube}. In light of this research focus, the promise of deep and thoughtful online discussion becomes more relevant. However, because irrelevant and low-quality comments frequently crowd and dilute meaningful online conversations \cite{CommentQuality, FlamingCommunications}, this promise remains unfulfilled.

We hypothesize that enhancing comment quality is only part of the challenge. Even high-quality comments have a limited influence on readers because of the collectively limited visibility of all comments. Commenting systems most commonly rest beneath, and detached from, the main content of modern websites. Compelling existing research has shown that this separation amplifies the reader's confusion of the comments~\cite{NewsInterfaces}. The reader must use more cognitive effort to match information referenced in comments to the primary content when there is greater distance between the information~\cite{FluidDocs}. This also encourages a cycle in which the reader is more likely to misinterpret or misreference the primary content, and comments become more error prone. This separation interferes with the transfer of information from the primary to secondary text~\cite{Brush,Guzdial,van}. It ultimately hinders the fluid facilitation of multi-reader discussions in the secondary content~\cite{AnnotationsStudents,NB}.

An alternative, explored chiefly in educational settings~\cite{Brush,Guzdial,van,AnnotationsStudents}, is to present comments on the margin of the primary text, visibly linked to relevant sections of the text. Such anchored discussions improve conversation among readers of academic texts by making the context of the conversation clearer~\cite{Brush,Guzdial,van}. A study, which examined the placement of student- and teacher-generated annotations on students' comprehension of a text, found that students who saw and used annotations adjacent to the main text displayed a deeper understanding of the text than those who interacted with annotations placed at the end of the text~\cite{AnnotationsStudents}.

We build on these insights to investigate whether placing user comments beside the primary text can encourage people to consider diverse perspectives when forming their individual opinions of an issue online. To enable an empirical investigation into this question, we have developed MindMargin, a system that exposes users to comments alongside the primary text as they read. The comments are anchored to relevant sections of the content in a minimally invasive design and are placed on an infinite horizontal scroll to reduce comment congestion.

%cluttering.

%Another study compares the four leading annotation interfaces: footnotes, interlinear commentary, ``sticky-note`` annotations, and marginal comments, concluding that the marginal interface is superior in minimizing distraction and enhancing visibility. The marginal interfaces studied, however, interact with the primary text. They are tagged to boxes around the referenced content to indicate where they are anchored~\cite{AnnotationsStudents}. 

%This structure parallels the relationship between content and comments, as interactions with comments are markedly secondary to the primary task of reading or browsing primary content. However, studies on Fluid Documents and annotation interfaces that challenge typographic conventions, such as discussion boards and footnotes, reveal limitations in the traditional vertical interface and suggest alternative horizontal layouts~\cite{Brush,Guzdial,van,AnnotationsStudents,NewsInterfaces,FluidDocs,NB}. 

%the comments section of many websites has become a popular space for “flame wars.” The social act of “flaming,” or the posting of offensive content, regularly devolves into hostile fights among multiple users, diverting a legitimate discussion topic to an unrelated and often emotionally charged digression. This behavior is examined closely in many research fields, including user interface design, communications, and psychology~\cite{FlamingPsych,FlamingCommunications,FlamingSoftware,FlamingComp}. 

%The Fluid Documents project aims to make information added into a page easier to locate in its source document by adjusting the typography of page. One specific study found that a fluid margin interface, as opposed to a fluid interline or overlay interface, had minimal disturbance to the user, because it did not move or occlude the primary text with the secondary text~\cite{FluidDocs}. 

%We have thus chosen to anchor comments in MindMargin with a faint dotted line to the edge of the article that corresponds to the y-coordinate of its reference, yet still avoids disturbing the primary text.

%Evidence in educational research, however, has shown that anchored annotated notes foster a deeper understanding of the text and facilitate more thoughtful teacher-to-student and peer-to-peer discussions~\cite{AnnotationsStudents,NB}. 

%Studies on balancing people’s skewed opinions also suggest exposing readers to a variety of relevant perspectives so that they consider views that exist beyond a single article and author’s scope~\cite{ConsiderIt,Politics,NewsCube}. 

%Thus, while current commenting platforms often fail at drawing people into sensible and relevant conversation, we offer evidence to suggest a horizontal interface such as MindMargin can change people’s views and actions, and thus address this societal problem. From our evaluation and user study, it appears that people with prior exposure to the issue in the article become more moderate in their opinions, reporting less polar views than those using a traditional vertical interface.

We propose two hypotheses for MindMargin's effect on users, in contrast to the traditional vertical interface:
\begin{enumerate}
\item Users of MindMargin will develop more thoughtful and nuanced opinions of an article than users using a traditional vertical interface who are presented with identical comments. % because MindMargin encourages readers to consider alternate views by exposing them to a greater diversity and number of comments.  
Specifically, we expect users who interact with MindMargin to report more moderate stances on controversial issues than users of a traditional vertical interface.
\item Users of MindMargin will report a more positive subjective impression of the existing comments because MindMargin displays anchored comments that appear alongside relevant passages of the text. We expect that comment placement will make the relevance of the comments to the primary text more apparent to users interacting with MindMargin.
\end{enumerate}

%\subsection{Contributions}

The results of our online study show that participants had significantly more positive impressions of the comments related to a controversial article when those comments were presented with MindMargin, as opposed to when those same comments were presented below the main article.  Participants who used MindMargin also reported a more moderate stance on the controversial issue raised in the article than participants who interacted with the traditional vertical layout interface.  Although this difference was not yet significant, it provides sufficient evidence for us to pursue a larger-scale study.


%This study presents the following novel contributions:
%\begin{itemize}
%\item A horizontally structured commenting interface such as MindMargin can expose users to a more diverse set of comments and compel those with existing exposure to the issue to think less divergently
%\item Anchored comments to relevant sections of the text in MindMargin compels users to not only have a more positive impression of, but also place greater trust in, the comments made by others
%%\begin{enumerate}
%%\item less extreme, polarized views on an issue by exposing them to a diversity of comments
%%\item a greater positive impression of comments made by other users 
%%\end{enumerate}
%\end{itemize}



%\section{Related Work}

Previous research has analyzed how different user interfaces facilitate facilitate the gathering of information and the digestion of articles. The \textit{Brussell} system uses a semantic model to provide references across various news articles~\cite{NewsInterfaces}. By highlighting and interacting with passages in the original article, the user can request additional background information from different sources. Another tool that supports collaborative document annotation within academic institutions is called \textit{NB} and focuses on honing users' understanding of lengthier articles~\cite{NB}. The authors report how students' comments in the NB system have impacted teachers to adapt their teaching style. One of the key features of NB is the geographic locality of annotations next to the article that helped facilitate student comprehension. This optimization of screen real-estate through a horizontal layout was also studied in connection with user engagement~\cite{AnnotationsStudents}. Students were found to reflect more critically on articles when comments were anchored on the side. This horizontal structure aligns with MindMargin's juxtaposition to its reference media. \textit{Opinion Space} presents another novel interface for commenting: a visualization technique that focuses on the diversity of aggregate comments rather than solely popular ones~\cite{OpinionSpace}. MindMargin addresses this issue as well by maintaining comments in traditional textual form to reduce any disturbance to the user's engagement. 

%Other studies have shown that it is possible to enhance community content contribution by providing subtle interfaces without disruptive character or radical representation changes~\cite{Wikipedia}. These interfaces stimulated the secondary task of editing Wikipedia articles rather than passive consumption as the primary task. Enhancing user engagement is also the topic of 

It is important to consider the distinction between engagement and such disturbance. When applied to designing user interfaces, active pop-up comments have been found to be highly useful to designers, but also somewhat disruptive to their concentration~\cite{CommentingSystems}. An interface that has too many moving elements may disturb the reader's focus more than engage them or supplement the material. However, some degree of disturbance is necessary for visibility. Because the active decision of commenting is a secondary task to the user's main goal of passively reading the article, users are more likely to engage in a secondary task if the threshold of engaging with that task is low. However, methods that lower barriers to the task most effectively, like pop-ups, are also highly disruptive~\cite{Wikipedia}. Improving comments must thus optimize for minimal disruption and maximal engagement.
Other studies have looked into ways to complement commenting systems. \textit{Reflect} is a platform that enables users to summarize each others comments in order to enhance comprehension and provide feedback on comments~\cite{Reflect}. \textit{Balancer} is a complementary widget that provides information on how politically skewed a user's aggregate source of news is~\cite{Politics}. These studies reveal that there exists a need in current commenting platforms to enhance engagement so that users better comprehend the reference media and consider the views of others.
 

\section {MindMargin}

\marginpar{
\begin{figure}
  \begin{center}
  \includegraphics[width=\marginparwidth]{mindmargin.png}
  \caption{The MindMargin system with the reference medium on the left and an adjacent commenting system on the right.}
  \label{fig:frontend}
  \end{center}
\end{figure}
}

\marginpar{
\begin{figure}
  \begin{center}
  \includegraphics[width=\marginparwidth]{traditional.png}
  \caption{The traditional commenting system with a vertically ordered design.}
  \label{fig:traditional}
  \end{center}
\end{figure}
}
We implemented two commenting systems. The first commenting system is MindMargin with anchored comments on a horizontal infinite scroll next to the reference medium (see Figure \ref{fig:frontend}). The second commenting system is a traditional vertical interface (see Figure \ref{fig:traditional}). %The two prototypes consist of clean user interfaces to avoid design clutter and distraction. 

MindMargin is split into two sides: the primary content on the left and and an adjacent commenting system on the right. Comments are dispalyed in a horizontal infinite scroll. Thus, an unrestricted amount of comments can be linked to the reference medium. Navigation within the infinite scroll component can be performed via mousewheel interaction (either left/right or top/down scrolling with the same effect) or by adjustment of a slider on the bottom of the right split screen. While navigating through the infinite scroll, the reference medium remains fixed on the left. Similar to~\cite{FluidDocs, NB}, comments are anchored to the horizontal reference point of the primary content. We minimize disturbance by avoiding interactions with the primary text, as defined by~\cite{FluidDocs}, and using thin dotted lines for anchoring to the article's right edge. This design decision was motivated by MindMargin's inherently more visible and thus distracting interface than the traditional vertical system's. 

If a comment has replies, a dropdown button appears on the comment's footer. Lighter in color, replies to comments appear vertically under their comment when the button is clicked. This arrangement optimizes horizontal real estate by reserving horizontal space for parent comments. Readers can upvote and downvote comments. Most recent comments and most popular comments are displayed directly next to the reference medium for increased visibility.
   
Our implementation of the traditional vertical commenting system follows a vertically ordered design: The primary content appears first and on top of the commenting system that follows below. Navigation within the article as well as within the comments can be performed via top/down scrolling. The replies, upvoting, and downvoting functions are organized in the same manner as MindMargin.

\section{Experiment}

We performed a between-subjects online experiments with young adults. %Participants were randomly assigned to one of two conditions: MindMargin or the vertical commenting interface.

\minisubsection{Participants}
106 online participants landed on our page for our user study and evaluation, of which 46 proceeded to begin and complete the study (30 female). 19 participants were randomly assigned to the Mind Margin condition and 27 to the vertical interface condition.  Participants were recruited online through social media and college listservs. Participants were college students, aged 18 to 25, and 68\% hailed from the local university. The self-reported reading frequency of online news among participants ranged from daily to almost never. 

\minisubsection{Experimental Conditions}
The two conditions in our study were MindMargin and the traditional vertical interface, both seeded with existing comments from a relevant news article. The article was selected on the basis of its opinionated nature and its relevance both in recent news and to our anticipated participant pool. We chose an opinion piece from our university's undergraduate publication, titled ``Don't Teach for America.'' Teach For America (TFA) is a non-profit organization that recruits recent college graduates to teach for two years in public schools. 

The article already had over fifty comments by affiliates and non-affiliates of the university alike, from which we selected the top 39 comments as ranked by Disqus, the existing commenting system on the publication's website, to be used in our study. The same comments were used in both conditions. In the traditional vertical interface, the comments appeared in the identical order as ranked in the original article. In MindMargin, we anchored them to the article based on textual references, specific phrases, quotes, and relevant content in each comment. 

%Users could make new comments by writing in the static new comment box above existing comments in the traditional interface and by clicking any part of the article to open a new comment box in MindMargin. In the MindMargin condition, we provided participants with simple, temporary instructions: "click the text to comment."

%XXXX include that last paragraph? XXXX

\minisubsection{Tasks}
To ensure that our results would be informative for the design of real-world commenting systems, we designed the experimental tasks to focus participants' attention on the content of the article.  The study design did not emphasize that the evaluation of the commenting system was the object of the study, but this information was clearly communicated and disclosed in the consent form. 

Participants were presented with an article and they were instructed to read the article and to anticipate a questionnaire that followed, but were not asked or required to interact with the commenting system.  Once they completed reading the article, we asked them the general stance of the article--For TFA or Against TFA--and, in a free-text response, we requested two pieces of supporting evidence used in the article to verify their reading and comprehension of the article. All 46 participants gave correct and thorough answers to these verification questions. Participants were then asked to complete a post-experiment questionnaire, which asked for their personal stance on the issue, whether they liked the article, and whether they agreed with the article. They were also asked to self-report whether they read the comments in the article and to provide two adjectives that described either their reaction to, or a description of, the comments.


%and were not permitted to refer back to the article once the questionnaire was administered. These questions included the following: [[XXX list questions in questionnaire -- KZG: or at least summarize the purpose of the questionnaire. Oh, here's a trick: you can make the questions into a figure and post them on the margin.]]

To further incentivize participants to focus on the content of the article and reflect on the issue discussed therein, we used the following tagline to advertise the study: ``Do you (really) think like a Harvard student?'' and at the completion of the study, we provided them with feedback comparing their answers to the responses made by other Harvard students.

%In order to reproduce the conditions under which one would normally read a news article, we chose to recruit participants online and to allow them to self-select themselves into reading the article based on personal time and interest. In order to motivate our participants to actually read or skim the article, instead of skip it, we chose not to use monetary or other time-sensitive incentives. Instead, we chose to design the experiment around the survey question, "Do you (really) think like a student [from our local university]?" 
%
%We then asked participants follow-up questions to verify that they read the article, which included both the overall stance of the article and, in a free-text response, two pieces of supporting evidence used in the article. All 46 participants gave correct and thorough answers to these verification questions. Participants were then asked to complete a post-experiment questionnaire and were not permitted to refer back to the article once the questionnaire was administered.

\minisubsection{Procedure}
Participants were given an initial questionnaire asking basic demographics and news reading frequency. Before being shown the article, they were also asked either to provide a username or pseudonym, or to remain anonymous. Participants were allotted 10 minutes to read the article. After 2 minutes, they were permitted to proceed to the questionnaire. The 2-minute delay was to ensure the reading of the article, and did not seem to prevent fast readers from moving too slowly, as the average reading time was 3 minutes 47 seconds. 

%In the follow-up questionnaire, reading verification questions were first posed. Participants were then asked their personal stance on the article, whether they liked the article, and whether they agreed with the article. They were also asked to self-report whether they read the comments in the article and to provide two adjectives that described either their reaction to, or a description of, the comments.

\minisubsection{Design and Analysis}
We used a between-subject full factorial design with two factors and the interaction between them.

\marginpar{
\begin{figure}
  \begin{center}
  \includegraphics[width=\marginparwidth]{familiarity.png}
  \caption{When using MindMargin, participants with prior exposure to the article reported less extreme stance.}
  \label{fig:leastsqs}
  \end{center}
\end{figure}
}

\marginpar{
\begin{figure}
  \begin{center}
  \includegraphics[width=\marginparwidth]{mosaicplot_revised6.png}
  \caption{When using MindMargin, the majority of participants described the comments as positive (3) rather than neutral (2) or negative (1).}
  \label{fig:mosaic}
  \end{center}
\end{figure}
}

The two factors were: {\it Prototype} \{``mindmargin`` for the new horizontal interface MindMargin or ``regular`` for the traditional vertical interface\}, and {\it Prior exposure to the article} \{``read`` if they have read or skimmed the article before, ``seen`` if they have seen the article but did not read or skim it, and ``none`` if they have neither read nor seen the article previously\}. To compare conditions, we excluded 9 participants who reported not to have read the comments (5 MindMargin). In addition to commenting system design, there was no significant difference in gender, reading frequency, or other demographic information among excluded participants.

We analyzed two dependent variables.  First, we computed {\it Stance Polarization}, which captured how far a participant's personal stance on the issue (on a scale from 0-Strongly For TFA to 100-Strongly Against TFA) was removed from the neutral stance (50).  We excluded 2 participants who opted out of answering this question (1 MindMargin), both of whom were already excluded for having not read the comments.

% after reading and interacting with their respective commenting systems. We computed each participant's SP, or their Stance Polarization, by measuring the distance between their stance and a neutral stance of 50 as $|50 - stance|$. 

Second, we measured {\it Attitude Toward Comments}.  As mentioned earlier, each participant who reported having read the comments was asked to provide two adjectives describing their reaction to the comments accompanying the article.
%For our second hypothesis, we examined the two free-text responses describing either participants' reaction to the comments or their description of the comments, both in response to the question: ``What did you think of the comments from the article?`` 
We asked four independent raters, blind to the experiment, to classify these adjectives using a four-bin classifier (``Positive,`` ``Negative,`` ``Neutral,`` and ``Invalid``). %They were told to classify the adjectives provided that they were in answer to the free-text question, posed on the survey: ``What did you think of the comments (from article X)?`` 
We removed adjectives given two or more ``Invalid`` classifications. % as outliers, which only appeared in the vertical condition. 
For the remaining adjectives, we used the majority vote.  Disagreements among raters were never between ``Positive'' and ``Negative''.  
%We found that adjectives without uniform encoding observed an uncontradictory mix of classifications: ``Positive`` and ``Neutral`` or ``Neutral`` and ``Negative,`` but never ``Positive`` and ``Negative``. 
We were thus confident in using the resulting median encodings for the final classification of the specified adjectives.

We treated Stance Polarization as a continuous variable and we analyzed it using analysis of variance.
We treated Attitude Toward Comments as an ordinal variable and we analyzed it using ordinal logistic regression.

%we examined  the adjectives that participants who read the comments provided in reaction to the comment

\section{Results and Discussion}

In this section, we report on the findings of our user study. Overall, we observed an increase in user engagement when comparing the proposed MindMargin interface against the traditional vertical commenting system. We were able to accept two of the three initially defined hypotheses.

Our first hypothesis predicted an overall increase in newly generated user comments, replies, upvotes, and downvotes when using MindMargin. We believed that having the comment system permanently visible on the side while reading an article, as in MindMargin, should increase these actions. However, this was not the case: we \textbf{rejected Hypothesis 1} since we did not observe an increase in the reading rate of comments or in interactions with comments when using our system. In fact, the participant group using MindMargin stated a 73\% reading rate of comments while the control group using the traditional interface stated a 85\% reading rate. In addition, there was no statistically significant increase in commenting rate, reply rate or the rate of up- and down-voting. We have several theories to explain these phenomena. First, we think that the users were overwhelmed by the number of existing comments. We populated both interfaces with 39 comments and replies, extracted from the original article. These comments covered a wide range of controversy regarding the content and participants might have thought that everything was already said. Second, since we were testing a rather unusual interface, we think that the novelty effect of introducing a new horizontal interface should not be underestimated. Third, because the study was hosted on an external site from that of the original article, which typically does not display news, participants may have concentrated more on reading the article than providing comments. Fourth, we do not know the number of comments that participants who reported to have read the comments did actually read.

Our second hypothesis was defined as an increase in personal reflection when using MindMargin. Exposure to a range of controversial comments should result in the rethinking and revising of one’s own opinions. All participants were asked their stance, from Strongly For TFA to Strongly Against TFA, on a Likert scale. Using data from participants who reported to have read the comments (see above), we observed the percentage of participants who claimed a strong stance on the article. Of those assigned to the MindMargin interface, only 16\% reported to be either Strongly For TFA or Strongly Against TFA. In contrast, 26\% of the participants using the traditional commenting system reported either extreme stance. The distribution of the Likert values is also normal for MindMargin and a U-shaped curve for the traditional commenting system. We performed a Shapiro-Wilk normality test ($alpha=0.05$) on both distributions. MindMargin rejects the null-hypothesis with $p=0.1306$ and therefore is normally distributed. The traditional prototype accepts the null-hypothesis with $p=0.0205$ and is therefore not normally distributed. We created Normal Q-Q plots for both (figure \ref{fig:mm_normal} and \ref{fig:reg_normal}). This reveals that despite no increase in the rate of reading comments, the MindMargin interface was able to encourage users to consider other opinions and viewpoints. This suggests a greater user engagement with the comments with the MindMargin interface. Therefore, we have \textbf{accepted Hypothesis 2}.

\begin{figure}
\centering
\includegraphics[scale=0.25]{mm_normal.png}
\caption{The personal stance distribution for MindMargin participants (normally distributed according to the Shapiro-Wilk normality test with $alpha=0.05$, $p=0.1306$}).
\label{fig:mm_normal}
\end{figure}

\begin{figure}
\centering
\includegraphics[scale=0.25]{regular_normal.png}
\caption{The personal stance distribution for traditional interface participants (normally distributed according to the Shapiro-Wilk normality test with $alpha=0.05$, $p=0.0205$}).
\label{fig:mm_normal}
\end{figure}

Our third hypothesis predicted an overall increase in positive impressions on comments when using the MindMargin interface. We asked participants who read the comments to input two adjectives in free-text describing either their reaction to the comments or a description of the comments. We then classified these adjectives using a three-bin classifier (“Positive,” “Negative,” and “Neutral”). “Positive” was assigned to positive reactions to comments, such as “interesting,” “well thought-out,” and “engaging.” “Negative” was assigned to negative reactions to comments, such as “annoying,” “useless,” “distracting.” “Neutral” was assigned to descriptive input about the comments, such as “long” and “subjective.” Finally, a few outliers, such as “trolls” and “whatever,” were removed. We observed a drastic change of impressions when using MindMargin. As seen in figure XXX, the majority of participants using the traditional commenting system described the comments as negative (68\%). In contrast, when using MindMargin, the majority of participants described the comments as positive (48\%) or neutral (48\%) as seen in figure YYY.  Outliers were also observed only to occur in the traditional commenting system. We have therefore \textbf{accepted Hypothesis~3}. 

The increased positive reaction to comments suggests that users perceive comments to be more substantial and worth greater consideration in the MindMargin interface. Even though participants were exposed to the identical comments and the comments were initially written for the traditional vertical interface, reactions to the comments were significantly more positive when using MindMargin. The rejection of Hypothesis 1 renders this finding even more surprising because there was no statistical difference in the number of people who read comments in either interface. The rejection ultimately supports the increase in user engagement with the MindMargin interface, and even suggests that MindMargin is not a distracting interface since participants can still opt out of reading comments if they so choose. Moreover, those who did not opt out of reading the comments found the exposure to anchored, relevant comments in MindMargin to be considerably more engaging and less distracting. In addition our quantitative results, we would like to quote qualitative feedback from a MindMargin user, suggesting actions he/she took beyond the scope of reading and commenting article: “This article showed me a new perspective on TFA, which after doing research, I have realized I agree with.” No feedback suggesting actions outside the scope of the article was received from participants with the traditional commenting system. 

\section{Conclusions}

In this paper, we studied how commenting systems can increase user engagement with comments while reading articles. We report quantitative evidence supporting MindMargin as an interface that depolarizes existing views and establishes new ones by exposing readers to diverse opinions in the comments and that enhances readers' overall impression of existing comments on the article. The key difference between traditional commenting systems and MindMargin is that in the latter, comments are anchored to specific passages of the reference media and are placed on a horizontal infinite scroll. We developed two commenting systems, one using the traditional vertical interface and the other using the MindMargin interface. 

Then, we performed a user study for evaluation. Our key findings include that being exposed to relevant comments during reading increases personal reflection. This results in 10\% less extreme positions regarding the context of the reference article. Additionally, the overall impression of comments significantly diverges. 68\% of users of the traditional commenting system report comments to be negative, while only 2\% of MindMargin users report comments to be negative. 

Future research will include a user study without already seeded comments as well as employ a within-subjects methodology. In addition, we plan to expand the participant pool to include participants of all ages and backgrounds. We would like to explore if MindMargin causes increased difficulty for readers to leave inflamed comments because they must choose an appropriate place to anchor their highly visible comment. Finally, we plan to pursue research on a commenting system like MindMargin, but for videos and music, that anchors comments to certain times or time-intervals within a given recording. Research into annotations on visual pieces, other than text, is also being considered.


%\section{References format}
%References must be the same font size as other body text.
% REFERENCES FORMAT
% References must be the same font size as other body text.

\balance
\bibliographystyle{acm-sigchi}
\bibliography{lit.bib}

\end{document}