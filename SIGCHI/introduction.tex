\section{Introduction}

News pages, media sites, online shops, blogs and social networks support the ability to give content-related feedback in the form of comments. Commenting systems on these sites are traditionally featured under, and separate from, the main content. This structure parallels the relationship between content and comments, as interactions with comments are markedly secondary to the primary task of reading or browsing primary content. However, studies on Fluid Documents and annotation interfaces that challenge typographic conventions, such as discussion boards and footnotes, reveal limitations in the traditional vertical interface and suggest alternative horizontal layouts~\cite{FluidDocs,NewsInterfaces,NB,AnnotationsStudents,Brush,Guzdial,van}. 

The Fluid Documents project aims to make information added into a page easier to locate in its source document by adjusting the typography of page. One specific study found that a fluid margin interface, as opposed to a fluid interline or overlay interface, had minimal disturbance to the user, because it did not move or occlude the primary text with the secondary text~\cite{FluidDocs}. 

The concept of anchoring annotations to references in a text has been known to qualitatively improve conversation among readers because it makes understanding the context of a comment cognitively easier~\cite{Brush,Guzdial,van}. Another study compares the four leading annotation interfaces: footnotes, interlinear commentary, ``sticky-note`` annotations, and marginal comments, concluding that the marginal interface is superior in minimizing distraction and enhancing visibility. The marginal interfaces studied, however, interact with the primary text. They are tagged to boxes around the referenced content to indicate where they are anchored~\cite{AnnotationsStudents}. 

We have thus chosen to anchor comments in MindMargin with a faint dotted line to the edge of the article that corresponds to the y-coordinate of its reference, yet still avoids disturbing the primary text.

While online commenting provides an opportunity for readers to express their views and engage in lively discussion with others, the comments section of many websites has become a popular space for “flame wars.” The social act of “flaming,” or the posting of offensive content, regularly devolves into hostile fights among multiple users, diverting a legitimate discussion topic to an unrelated and often emotionally charged digression. This behavior is examined closely in many research fields, including user interface design, communications, and psychology~\cite{FlamingComp,FlamingPsych,FlamingSoftware,FlamingCommunications}. 

Evidence in educational research, however, has shown that anchored annotated notes foster a deeper understanding of the text and facilitate more thoughtful teacher-to-student and peer-to-peer discussions~\cite{AnnotationsStudents,NB}. Studies on balancing people’s skewed opinions also suggest exposing readers to a variety of relevant perspectives so that they consider views that exist beyond a single article and author’s scope~\cite{Politics,NewsCube,ConsiderIt}. 

Thus, while current commenting platforms often fail at drawing people into sensible and relevant conversation, we offer evidence to suggest a horizontal interface such as MindMargin can change people’s views and actions, and thus address this societal problem. From our evaluation and user study, it appears that people with prior exposure to the issue in the article become more moderate in their opinions, reporting less polar views than those using a traditional vertical interface.

\subsection{Hypotheses}
We propose two hypotheses for MindMargin's effect on its users in comparison to the traditional vertical interface:
\begin{enumerate}
\item Users of MindMargin will develop more thoughtful and nuanced opinions of an article, because MindMargin encourages readers to consider alternate views by exposing them to a greater diversity and number of comments.
\item Users of MindMargin will report a more positive impression of the existing comments because MindMargin displays anchored comments that appear alongside relevant passages of the text.
\end{enumerate}

\subsection{Contributions}
This study presents the following novel contributions:
\begin{itemize}
\item A horizontally structured user interface for anchored comments on websites
\item Insights into how comments can challenge readers' perspectives through
\begin{enumerate}
\item Exposure to a diversity of comments
\item Exposure to relevant comments at specific textual locations
\end{enumerate}
\end{itemize}

