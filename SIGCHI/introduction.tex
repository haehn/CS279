\section{Introduction}

\subsection{Motivation}
Many websites support the ability to give content-related feedback in the form of comments. Such websites include news pages, media sites, online shops, blogs and social networks. Both adding new comments and reading existing ones can engage the user in information exchange, personal reflection and lively discussion. To the reader, comments offer additional and corrective evidence as well as alternative views. Because of their diversity, comments have the ability to impact the reader’s opinions and motivate the reader to perform actions beyond the scope of the article, such as sharing the article, voting, purchasing a product, making choices, and participating in a cause. 

The traditional format of articles online presents comments beneath their associative content. This may seem appropriate at first because interactions with an article’s comments are secondary to the reader’s primary task of reading the article itself. However, this vertical structure impedes the reader's ability to view comments alongside the reference media. In fact, the traditional format forces readers to pass the entirety of the content before reaching the comments section. Commenters must either complete the primary task of reading the article or bypass parts of the article to reach the same comments. Even in multi-page articles such as those on The New York Times, the commenting system sits beneath each page, but includes comments from the entire article, not its associative page. Separate from the content, the traditional commenting system lacks a method of referencing specific sections within an article.

We present \textbf{MindMargin}, a commenting system that motivates people to view the world differently. MindMargin most notably has a horizontal commenting interface that appears adjacent to the article and allows for direct textual reference through comment anchoring. From our initial evaluation, we propose that this alternative commenting system can challenge a reader’s existing opinions by presenting them with relevant comments at appropriate points in the article. 

\subsection{Related Works}
\textbf{Existing Annotation and Commenting Systems}
Previous research on annotation and commenting systems has focused primarily on enhancing reader comprehension~\cite{NewsInterfaces}~\cite{Reflect}~\cite{NB}~\cite{AnnotationsStudents}. The systems proposed in these studies are able to accomplish this through improving commenter communication. For example, the Brussell system~\cite{NewsInterfaces} gathers information across news platforms and online references in order to answer reader-generated questions to unfamiliar or confusing phrases in online news articles. The Reflect system~\cite{Reflect} seeks to reproduce the “listening” component of offline conversation in the online setting by allowing readers to summarize and provide feedback on other readers’ comments. The NB PDF Annotation Tool~\cite{NB} targets teacher-student and student-student communication in the classroom by providing a means to annotate specific portions of a text to generate discussion. Another study~\cite{AnnotationsStudents} also examines textual annotation in an academic setting, in which students who are provided with annotations from their instructor and peers demonstrate more critical thinking, and less superficial summarization, than those who were not exposed to the system. Research on improving facilitation of communication in all the above systems suggests that stand-alone articles are limited in scope and that readers can reap much more intellectually from texts when presented with outside views and resources alongside their reading.

There have also been myriad studies on the disturbance-invisibility tradeoff of annotation and commenting systems as well as the effectiveness of anchoring annotations~\cite{FluidDocs}~\cite{Wikipedia}~\cite{AnnotationsStudents}. ~\cite{FluidDocs} defines disturbance as moving or occluding the article, or the primary text, with secondary text, or comments. The study shows that increased disturbance also leads to increased guidance and aid to the user, suggesting that close anchoring can be simultaneously effective and distracting. This is supported by ~\cite{Wikipedia} that concludes that popup interfaces, while effective in attracting user attention, ultimately prove ineffective in retaining the user to return to the site and motivating the user to engage in the secondary task of commenting on an article.  The academic annotation study ~\cite{AnnotationsStudents} notes that effective textual conversations take place in “annotation-based discussion environments.” However, the study also points out the difficulty in designing a usable and effective interface for annotation, exploring the four leading annotation interfaces, including footnotes, interlinear commentary, “sticky-note” annotations, and marginal comments, and concluding that the marginal interface is superior in minimizing distraction and enhancing visibility. The study continues to examine existing marginal interfaces, such as an infinite expanding margin with an annotator occupying a column of space, a set of expanding text boxes, and the traditional fixed single-column margin. These annotation interfaces displayed boxes around the reference text to indicate where it was anchored. Combining these ideas in an effort to maximize visibility while minimizing disturbance, we have chosen to anchor MindMargin comments with a faint dotted line corresponding to the y-coordinate of their textual reference, but avoiding any disturbance on the the article content. 

\textbf{Exposure to Diverse Perspectives}
Politically charged studies on user interface, not specific to commenting systems, also emphasize the significance of considering views that exist beyond an article’s scope~\cite{Politics}~\cite{NewsCube}~\cite{ConsiderIt}. The Balancer system~\cite{Politics} indicates to readers of online news how politically skewed their source of news is based on their choice of news outlets, and recommends other sources to counterbalance the bias in their reading. NewsCube~\cite{NewsCube} proposes a similar recommending system by classifying aspects, suggesting articles with contrasting aspects, and allowing readers to browse by aspect. ConsiderIt~\cite{ConsiderIt} presents a pros and cons board for a variety of political issues, on which users can list their opinions. Because the board displays both pros and cons, including an equal length of pre-populated ones on the side, users are more inclined to balance their lists with both pros and cons. With access to a wider variety of perspectives, and presented in a manner that encourages political balance, readers are urged to consider alternate opinions and as a result, to adapt, strengthen, or clarify their own. Ultimately, both ConsiderIt and Balancer aim to depolarize readers’ opinions by exposing them to diverse perspectives.

As aforementioned, ~\cite{AnnotationsStudents} introduces students to other points of view through peer and teacher annotations anchored to the text. The study reveals that exposure to external, divergent thinking leads to more “critical responses that reflect active, independent thinking.” Critical analysis and active thinking were found to aid in shaping the opinions of readers with both within-subjects and between-subjects methodologies. The study also shows that this independence in thinking among novice readers, in particular, manifests itself in the strong opinions these readers develop when exposed to the annotation system. While ~\cite{AnnotationsStudents} studies the comparison between the existence and the absence of an annotation system that offers alternate perspectives, our research seeks to identify how the placement of such a system--horizontally or vertically--impacts and challenges readers’ views through the increased exposure to diverse viewpoints.

\subsection{Hypotheses}
We test MindMargin against the vertical commenting interface, and propose two hypotheses regarding their comparison:
\begin{enumerate}
\item MindMargin will encourage readers to define or refine their opinions, prompting new readers to develop a stance on an issue and encouraging readers with existing views to consider alternate views. MindMargin will accomplish this by exposing readers to a greater number of opinions and consequently compelling readers to think more independently and sharpen their opinions. 
\item MindMargin will compel readers to have a more positive impression of the comments by displaying comments anchored to relevant references in the text.
\end{enumerate}

\subsection{Contributions}
This project includes the following novel contributions:
\begin{itemize}
\item A horizontally structured user interface for anchored comments on websites
\item Insights into how comments can challenge readers’ perspectives through
\begin{enumerate}
\item Exposure to a diversity of comments
\item Exposure to relevant comments at specific textual locations
\end{enumerate}
\end{itemize}

