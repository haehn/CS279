\section{Introduction}

News pages, media sites, online shops, blogs and social networks support the ability to give content-related feedback in the form of comments. For readers and users, online commenting holds a compelling promise: to share their views, to contribute their relevant knowledge, and to engage in lively, thoughtful discussion with each other. As online political debate and discourse increasingly polarize and generalize individual political views, the prevailing opinion among researchers has been to present people with diverse perspectives in order to encourage more nuanced, independently reached points of view~\cite{ConsiderIt,Politics,NewsCube}. In light of this, the promise of deep and thoughtful online discussion becomes particularly valuable. However, because irrelevant and low-quality comments frequently crowd and dilute meaningful online conversations \cite{CommentQuality, FlamingCommunications}, this promise remains unfulfilled.

We hypothesize that enhancing comment quality is only part of the challenge. Even high-quality comments have a limited influence on readers because of their equally limited visibility as other comments. Commenting systems most commonly rest beneath, and detached from, the main content of modern websites. Compelling existing research, however, suggests that separating secondary or additional content, such as comments, from the primary content has lead to a decreased understanding of the primary content, a more superficial and less critical engagement with the primary content, an amplified confusion of the secondary content, increased errors of comprehension and proper reference in secondary content, increased cognitive difficulty to the reader in matching references in the secondary content to the primary content, ultimately interfering with the transfer of information from the primary to the secondary content and with the facility of fluid discussion in the secondary content~\cite{Brush,Guzdial,van,AnnotationsStudents,NewsInterfaces,FluidDocs,NB}.

An alternative that has been markedly explored in educational settings~\cite{Brush,Guzdial,van,AnnotationsStudents} is to present comments on the margin of the primary text, visibly linked to relevant sections of the text. Such anchored discussions improve conversation among readers of academic texts by making the context of the conversation clearer~\cite{Brush,Guzdial,van}. A study, which examined the placement of student- and teacher-generated annotations on students' comprehension of a text, found that students who saw and used annotations adjacent to the main text displayed a deeper understanding of the text than those who interacted with annotations placed at the end of the text~\cite{AnnotationsStudents}.

We build on these insights to investigate whether placing user comments beside the primary text can encourage people to consider diverse perspectives when forming their individual opinions of an issue online. To enable an empirical investigation into this question, we have developed MindMargin, a system that exposes users to comments alongside the primary text as they read. The comments are anchored to relevant sections of the content in a minimally invasive design and are placed on an infinite horizontal scroll to reduce comment congestion.

%cluttering.

%Another study compares the four leading annotation interfaces: footnotes, interlinear commentary, ``sticky-note`` annotations, and marginal comments, concluding that the marginal interface is superior in minimizing distraction and enhancing visibility. The marginal interfaces studied, however, interact with the primary text. They are tagged to boxes around the referenced content to indicate where they are anchored~\cite{AnnotationsStudents}. 

%This structure parallels the relationship between content and comments, as interactions with comments are markedly secondary to the primary task of reading or browsing primary content. However, studies on Fluid Documents and annotation interfaces that challenge typographic conventions, such as discussion boards and footnotes, reveal limitations in the traditional vertical interface and suggest alternative horizontal layouts~\cite{Brush,Guzdial,van,AnnotationsStudents,NewsInterfaces,FluidDocs,NB}. 

%the comments section of many websites has become a popular space for “flame wars.” The social act of “flaming,” or the posting of offensive content, regularly devolves into hostile fights among multiple users, diverting a legitimate discussion topic to an unrelated and often emotionally charged digression. This behavior is examined closely in many research fields, including user interface design, communications, and psychology~\cite{FlamingPsych,FlamingCommunications,FlamingSoftware,FlamingComp}. 

%The Fluid Documents project aims to make information added into a page easier to locate in its source document by adjusting the typography of page. One specific study found that a fluid margin interface, as opposed to a fluid interline or overlay interface, had minimal disturbance to the user, because it did not move or occlude the primary text with the secondary text~\cite{FluidDocs}. 

%We have thus chosen to anchor comments in MindMargin with a faint dotted line to the edge of the article that corresponds to the y-coordinate of its reference, yet still avoids disturbing the primary text.

%Evidence in educational research, however, has shown that anchored annotated notes foster a deeper understanding of the text and facilitate more thoughtful teacher-to-student and peer-to-peer discussions~\cite{AnnotationsStudents,NB}. 

%Studies on balancing people’s skewed opinions also suggest exposing readers to a variety of relevant perspectives so that they consider views that exist beyond a single article and author’s scope~\cite{ConsiderIt,Politics,NewsCube}. 

%Thus, while current commenting platforms often fail at drawing people into sensible and relevant conversation, we offer evidence to suggest a horizontal interface such as MindMargin can change people’s views and actions, and thus address this societal problem. From our evaluation and user study, it appears that people with prior exposure to the issue in the article become more moderate in their opinions, reporting less polar views than those using a traditional vertical interface.

We propose two hypotheses for MindMargin's effect on users, in contrast to the traditional vertical interface:
\begin{enumerate}
\item Users of MindMargin will develop more thoughtful and nuanced opinions of an article than users using a traditional vertical interface who are presented with identical comments. % because MindMargin encourages readers to consider alternate views by exposing them to a greater diversity and number of comments.  
Specifically, we expect users who interact with MindMargin to report more moderate stances on controversial issues than users of a traditional vertical interface.
\item Users of MindMargin will report a more positive subjective impression of the existing comments because MindMargin displays anchored comments that appear alongside relevant passages of the text. We expect that comment placement will make the relevance of the comments to the primary text more apparent to users interacting with MindMargin.
\end{enumerate}

%\subsection{Contributions}

The results of our online study show that participants had significantly more positive impressions of the comments related to a controversial article when those comments were presented with MindMargin, as opposed to when those same comments were presented below the main article.  Participants who used MindMargin also reported a more moderate stance on the controversial issue raised in the article than participants who interacted with the traditional vertical layout interface.  Although this difference was not yet significant, it provides sufficient evidence for us to pursue a larger-scale study.


%This study presents the following novel contributions:
%\begin{itemize}
%\item A horizontally structured commenting interface such as MindMargin can expose users to a more diverse set of comments and compel those with existing exposure to the issue to think less divergently
%\item Anchored comments to relevant sections of the text in MindMargin compels users to not only have a more positive impression of, but also place greater trust in, the comments made by others
%%\begin{enumerate}
%%\item less extreme, polarized views on an issue by exposing them to a diversity of comments
%%\item a greater positive impression of comments made by other users 
%%\end{enumerate}
%\end{itemize}

