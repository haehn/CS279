\section{Results}

%\subsection{Hypothesis 1}
%Our first hypothesis predicted that MindMargin would have an impact on individual opinions, prompting new readers to develop a stance on the issue and encouraging readers with existing views to consider alternate views. The reasoning behind this hypothesis was that MindMargin exposes readers to a greater number of opinions while reading the article. Consequently, readers think more independently, reconsidering their own views in light of others', and develop more thoughtful and nuanced opinions. 

%All participants were asked their stance on a Likert scale from Strongly For TFA to Strongly Against TFA. In our analysis, we excluded data from participants who reported to have not read the comments (there was statistically no difference in stance between prototypes Num dropped=XX and Nums for ea. avg stance=XX). We also excluded participants who did not toggle the Likert scale and answer this question (Num==XX). The data of the remaining participants (N=XX) was remapped for their ``Stance Polarization,`` or the deviation of their stance from neutral, as $|50 - stance|$ (range 0-50). 

%Participants also reported on their familiarity with the article, which ranged from "read" if they had read or skimmed the article, to "seen" if they had seen the article in the past but did not read or skim it, to "none" if they had never encountered the article prior to this study. We then performed an analysis for both prototypes that included familiarity with the article as a covarying factor and the dependent variable SP, their ``Stance Polarization.`` 

%We found that for users whose familiarity of the article was ``none,`` there was no significant difference in the polarization of their stance between MindMargin and the traditional. For participants who were previously exposed to the issue in the article, and whose familiarity with the article was thus ``seen`` or ``read,`` we did in fact observe a difference in stance polarization between the two prototypes (see Figure YYYY). Participants whose familiarity of the article was ``seen`` had a SP value of 7 with MindMargin and 17 with the traditional system. Participants whose familiarity of the article was ``read`` had a SP value of 17 with MindMargin and 27 with the traditional system. 

People who used MindMargin reported a less extreme SP ($m=16.9$). People who used a traditional vertical interface reported a more extreme SP ($m=19.1$). However, this difference was not statistically significant ($F_{1,40}=1.16$, $p=.29$). We also tested for interaction between commenting system design and prior exposure to the article, and found that people who had prior exposure to the article had a more moderate stance using MindMargin (read $m=17.3$, seen $m=10.1$) than the traditional vertical interface (read $m=22.1$, seen $m=23.2$). This effect is illustrated in Figure \ref{fig:leastsqs}. 

%\subsection{Hypothesis 2}
%Our second hypothesis predicted an overall increase in positive impressions on comments when using the MindMargin interface. We asked participants who read the comments to input two adjectives in free-text describing either their reaction to the comments or a description of the comments. We then asked four independent volunteers, blind to the experiment, to classify these adjectives using a four-bin classifier (``Positive,`` ``Negative,`` ``Neutral,`` and ``Invalid``). They were told to classify the adjectives provided that they were in answer to the question, posed on the survey: ``What did you think of the comments (from article X)?`` 

%We removed adjectives given two or more ``Invalid`` classifications as outliers, which only appeared in the vertical condition. We found that adjectives without uniform encoding observed an uncontradictory mix of classifications: ``Positive`` and ``Neutral`` or ``Neutral`` and ``Negative,`` but never ``Positive`` and ``Negative``. We were thus confident in using the resulting median encodings for the final classification of the specified adjectives. 

%As seen in figure \ref{fig:mosaic}, the greater majority of comment impressions in the traditional interface was negative (negative at 34.8\%, neutral at 17.2\%, positive 4.7\%) while most of the comment impressions among users of MindMargin were classified either positive or neutral (negative at 4.7\%, neutral at 20.3\%, positive 18.8\%). 

As illustrated in Figure \ref{fig:mosaic}, we observed a significant main effect of Prototype on ATC ($\chi^2_{1,N=64}=16.55$, $p<.0001$).
People who used MindMargin had a more positive than negative impression of comments (positive: 42.86\%, negative: 10.71\%, neutral/descriptive: 46.43\%). People who used a traditional vertical interface had a more negative than positive impression of comments (positive: 8.33\%, negative: 61.11\%, neutral/descriptive: 30.56\%). There was no significant or substantial effect of prior exposure to the article on ATC.  %We observed a significant main effect of commenting system design on the impressions. This effect is illustrated in Figure \ref{fig:mosaic}. We also tested for interaction between commenting system design and prior exposure to the article XXX. 

%subjective responses == we observed a sig main effect of design of ppls subjective response. ppl in mm found to be blah (stats in parentheses). ppl in trad blah (stats). 