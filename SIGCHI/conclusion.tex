\section{Conclusions}

In this paper, we studied how commenting systems can increase user engagement with comments while reading articles. We report quantitative evidence supporting MindMargin as an interface that depolarizes existing views and establishes new ones by exposing readers to diverse opinions in the comments and that enhances readers' overall impression of existing comments on the article. The key difference between traditional commenting systems and MindMargin is that in the latter, comments are anchored to specific passages of the reference media and are placed on a horizontal infinite scroll. We developed two commenting systems, one using the traditional vertical interface and the other using the MindMargin interface. 

Then, we performed a user study for evaluation. Our key findings include that being exposed to relevant comments during reading increases personal reflection. This results in 10\% less extreme positions regarding the context of the reference article. Additionally, the overall impression of comments significantly diverges. 68\% of users of the traditional commenting system report comments to be negative, while only 2\% of MindMargin users report comments to be negative. 

Future research will include a user study without already seeded comments as well as employ a within-subjects methodology. In addition, we plan to expand the participant pool to include participants of all ages and backgrounds. We would like to explore if MindMargin causes increased difficulty for readers to leave inflamed comments because they must choose an appropriate place to anchor their highly visible comment. Finally, we plan to pursue research on a commenting system like MindMargin, but for videos and music, that anchors comments to certain times or time-intervals within a given recording. Research into annotations on visual pieces, other than text, is also being considered.
