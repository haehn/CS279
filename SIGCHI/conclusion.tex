\section{Conclusion}

In this paper, we studied how commenting systems can increase user engagement with comments while reading articles. We report a correlation between the novel MindMargin interface and increased personal reflection as well as a more positive overall impression of existing comments on the article. The key difference between traditional commenting systems and MindMargin is that in the latter, comments are anchored to specific passages of the reference media and are placed on a horizontal infinite scroll. We developed two commenting systems, one using the traditional vertical interface and the other using the MindMargin interface. 

Then, we performed a user study for evaluation. Our key findings include that being exposed to relevant comments during reading increases personal reflection. This results in 10\% less extreme positions regarding the context of the reference article. Additionally, the overall impression of comments significantly diverges. Users of a traditional commenting system report 68\% of comments to be negative, while users of MindMargin report only 2\% of comments to be negative. Contrary to our initial hypothesis, MindMargin does not generate more comments or more replies. We think that this is primarily due to a) the novelty effect and b) the existence of previously seeded comments. 

Therefore, future research will include a user study without already seeded comments. In addition, we plan to expand the participant pool to include participants of all ages. Finally, we plan to pursue research on a commenting system like MindMargin, but for videos and music, that anchors comments to certain times or time-intervals within a given recording. Research into annotations on visual pieces, other than text, is also being considered.
