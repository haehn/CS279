\section{Discussion}

The lower SP values among MindMargin users reveal that participants with MindMargin who had prior exposure to the article reported less polarized views to reading the article for a second read or glance. MindMargin users also had a significantly more positive impression of the comments. 

The implications of these results are substantial. They suggest that MindMargin exposes readers to comments when they are most relevant to the reader, making the comments seem more engaging, thoughtful, and reliable. Since all participants were exposed to identical comments, we conclude that MindMargin readers consider the comments more substantial and ultimately consider others' opinions more seriously. Through greater exposure to diverse perspectives, MindMargin also motivates readers who had a prior opinion of the issue to consider opposing perspectives. As a result of this consideration, they are compelled to revise their own views of an issue to be more moderate. This is particularly compelling in light of the growing number of irrelevant, low-quality, and insult-ridden discussions, in which people's opinions are not taken seriously, on existing commenting systems. Because MindMargin  forces readers to choose an appropriate anchoring place for their comments, it also makes it more difficult to leave inflamed comments. 

However, there are limitations to our observations, since the results of our first hypothesis indicate that our trends are not yet statistically significant. This is chiefly a result of our sample size being not large enough. Another reason is the lack of a preliminary test in choosing a between-subjects methodology for our experiment. To address these limitations in subsequent research, we aim to expand the participant pool both in number and diversity. Additionally, we plan to use a within-subjects methodology, where we will ask participants for their stance on the article's issue prior to their reading of the article. We will subsequently observe the deltas in SP values.

In addition to these results, a MindMargin user gave us written feedback on their experience that suggests he/she took actions beyond the scope of reading and commenting article:``This article showed me a new perspective on TFA, which after doing research, I have realized I agree with.`` No similar feedback was received from participants with the traditional commenting system. While this is not enough to conclude that MindMargin motivated the participant to research the subject further, this nevertheless indicates that he/she, when exposed to the MindMargin interface, thought critically and independently about the issue discussed in the article.