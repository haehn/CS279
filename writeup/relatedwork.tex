\section{Related Work}

Previous research has analyzed how different user interfaces facilitate the gathering of information and the digestion of articles. 

The \textit{Brussell} system uses a semantic model to provide references across various news articles~\cite{NewsInterfaces}. By highlighting and interacting with passages in the original article, the user can request additional background information from different sources. 
%REFERENCES->COMPREHENSION

Another tool that supports collaborative document annotation within academic institutions is called \textit{NB} and focuses on honing users' understanding of lengthier articles~\cite{NB}. The authors report how students' comments in the NB system have impacted teachers to adapt their teaching style. One of the key features of NB is the geographic locality of annotations next to the article that helped facilitate student comprehension. %STUDENT COMPREHENSION AND ATTENTION

This optimization of screen real-estate through a horizontal layout was also studied in connection with user engagement~\cite{AnnotationsStudents}. Students were found to reflect more critically on articles when comments were anchored on the side. This horizontal structure aligns with MindMargin's juxtaposition to its reference media. %RELECTION,CRITICAL ANALYSIS

\textit{Opinion Space} presents another novel interface for commenting: a visualization technique that focuses on the diversity of aggregate comments rather than solely popular ones~\cite{OpinionSpace}. 
%DIVERSITY OF POVs

%MindMargin addresses this issue as well by maintaining comments in traditional textual form to reduce any disturbance to the user's engagement. 

%Other studies have shown that it is possible to enhance community content contribution by providing subtle interfaces without disruptive character or radical representation changes~\cite{Wikipedia}. These interfaces stimulated the secondary task of editing Wikipedia articles rather than passive consumption as the primary task. Enhancing user engagement is also the topic of 

It is important to consider the distinction between engagement and disturbance. When applied to designing user interfaces, active pop-up comments have been found to be highly useful to designers, but also somewhat disruptive to their concentration~\cite{CommentingSystems}. An interface that has too many moving elements may disturb the reader's focus more than engage him/her or supplement the material. However, some degree of disturbance is necessary for visibility. Because the active decision of commenting is a secondary task to the user's main goal of passively reading the article, users are more likely to engage in a secondary task if the threshold of engagement is low. However, methods that lower barriers to the task most effectively, like pop-ups, are also highly disruptive~\cite{Wikipedia}. Improving comments must thus optimize for minimal disruption and maximal engagement. %DISRUPTION THROUGH POPUPS BAD -> BALANCE ENGAGEMENT AND VISIBILITY

Other studies have looked into ways to complement commenting systems. \textit{Reflect} is a platform that enables users to summarize each others comments in order to enhance comprehension and provide feedback on comments~\cite{Reflect}. 
%MUTUAL COMPREHENSION / CHECKING OTHER USERS' COMPREHENSION

\textit{Balancer} is a complementary widget that provides information on how politically skewed a user's aggregate source of news is~\cite{Politics}. These studies reveal that there exists a need in current commenting platforms to enhance engagement so that users better comprehend the reference media and consider the views of others.
%DIVERSITY OF POVs and ENCOURAGE OTHERS' VIEWS
