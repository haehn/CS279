\section{Related Work}

Previous research has analyzed how different user interfaces facilitate the gathering of information and the digestion of articles. We compare related work in the field that has motivated our study and approach to the design of our system.
 
\textit{Horizontal Structure and Locality}
Having comments appear alongside a text has aided students’ reading comprehension and encouraged more critical analysis of readings in the classroom. \textit{NB} is a tool that supports collaborative annotation on PDFs~\cite{NB}. It allows both students and teachers to make comments alongside the text in the appropriate context. This results in less error in the students’ comprehension and draws the students’ attention to relevant comments. It also aims to facilitate clearer information exchange in the context of the article. NB improves the clarity and comprehension of lengthier articles in particular. 

While the comments on NB appear next to the article, another platform uses anchored comments that highlight the referenced section of the text~\cite{AnnotationsStudents}. This system was also studied with student users who were reported to reflect more critically on the the text with the tool and to be better prepared in argumentation activities. Students also had both qualitatively richer and quantitatively more discussions.

Apart from research on academic instruction, the \textit{Brussell} system uses anchored information in news articles~\cite{NewsInterfaces}. While not a commenting system, Brussell gathers information across various news sites as a means to provide accurate references and background information to parts of the the article unclear to the reader. By highlighting and interacting with passages in the original article, the reader can request additional background information from different sources. Brussell also suggests questions, such as "Who is this person?", for the reader to pose that probe deeper into the understanding of the article.

The geographic location of comments has nontrivial impact on the reader’s quality of understanding and consequent engagement of the article. As opposed to traditional vertical structures, a horizontal design provides comments that reference the text either directly or in the appropriate context. The optimization of screen real estate through a horizontally planted commenting system motivates our design for MindMargin.


\textit{Skewed and Confused: What’s Right?}
Determining what is important and what is extraneous in a comment is a challenge that all commenting systems must face. Whereas comment placement, as explored previously, affects the overall quality of comments, we must also consider how to increase the visibility of a diverse set of relevant comments. 

\textit{Opinion Space} presents a novel interface as a solution for diversifying comments in a system~\cite{OpinionSpace}. Instead of the traditional list format, OpinionSpace employs a visualization technique that reveals the diversity of aggregate comments. Thus, comments are ordered in any particular way and not shown to the reader with the oldest first, the most popular first, or in another skewed manner as is inevitable with a list. MindMargin’s horizontal structure naturally optimizes screen real estate and prevents clustering of comments by allowing them to spread along the length of the text. 

Because opinions in political news articles are frequently biased, another tool, called \textit{Balancer}, accompanies the reader on news articles and provides information on how politically skewed his/her aggregate source of news is~\cite{Politics}. This widget seeks to engage readers in a diverse set of perspectives. 

Both OpinionSpace and Balancer reveal that there exists value, as well as a need, in encouraging the consideration and discussion of divergent views online. However, the latter study also points out that people often gravitate to other people and articles with similar views or views that agree with their own. Thus, comments that disagree with the article are likely to be downvoted by the community and become invisible, further discouraging individuals with different views to read the article. In MindMargin, we aim to address this problem by developing a more objective voting system that upvotes a comment based on a diverse set of criteria, rather than a single thumbs-up vote of popularity. 

Another study examines the confusion that some even enriching comments present. \textit{Reflect} is a platform that enables users to summarize each others comments in order to enhance comprehension and provide feedback to fellow commenters~\cite{Reflect}. This aims to dissipate the confusion and extract the useful information in a comment. With anchored comments alongside the text, MindMargin is designed to minimize confusion by having readers reference the text directly and other readers have context to posted comments.


\textit{Invisibility-Disturbance Tradeoff}
In optimizing for comment engagement, we must minimize the disturbance that comments cause to the reader as well as maximize their visibility to ensure the reader’s engagement with them. Thus, there is a tradeoff between how hidden and how disruptive a comment is to the user’s reading experience. 

With traditional vertical platforms, comments are far from the view of the reader as he/she reads the article. They minimize disturbance but also, considering a reader’s primary task is to read the article, decrease the chance that he/she will read the comments. These platforms also split the engagement of reading an article from the engagement of reading the comments. This gap encourages some readers to skip some or all of the article and to depend on the comments alone for their source of knowledge from the article. In MindMargin, we seek to bridge this division with the horizontal commenting system. Readers are then encouraged to view comments as a companion to the article as opposed to an occupant of a separate niche.

While comments must be visible to draw and enhance reader engagement, highly visible comments can cause disturbance and instead decrease engagement. When applied to designing user interfaces, active pop-up comments have been found to be highly useful to designers, but also somewhat disruptive to their concentration~\cite{CommentingSystems}. An interface that has too many moving elements can disturb the reader's focus more than engage him/her in the material. 

Encouraging readers to comment also hinges on the visibility of the commenting system as well as its disturbance to the reader. The reader’s main goal and primary task are generally to read the article passively. The active decision of commenting on the article must be viewed as secondary. One study showed that readers are more likely to engage in a secondary task if the threshold of that secondary engagement is low~\cite{Wikipedia}. However, the methods that lower barriers to the task most effectively, like pop-ups, are also highly disruptive. In the design of MindMargin, the horizontal structure inherently adds to the visibility and disturbance of the system when compared against a vertical platform. Thus, to optimize for engagement, we must focus on actively minimizing the system’s disruption.