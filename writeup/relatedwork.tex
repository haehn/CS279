\section{Related Work}

Previous research has analyzed how different user interfaces facilitate information gathering and digestion of articles. The \textit{Brussell} system uses a semantic model to provide references across news articles~\cite{NewsInterfaces}. The user can request further background information from different sources by interacting with passages in the original article. Focusing on passages of a larger articles is also the main purpose of \textit{NB} - a tool for collaborative document annotation within academic institutions~\cite{NB}. The authors report how students' comments in the NB system have impacted teaching style adaptation. One of the key features of NB is the geographic locality of annotations next to the article. This optimization of screen real-estate through a horizontal layout was also studied in connection with user engagement~\cite{AnnotationsStudents}. Students were found to reflect more critically on articles when comments were anchored on the side. This is similar to our plans of a horizontal arrangement of comments regarding reference media. 
Another new interface for comments was proposed as \textit{Opinion Space}~\cite{OpinionSpace}. The authors developed a visualization technique focusing on a diversity of comments rather than only popular ones. We address this issue with MindMargin as well -- while maintaining comments in traditional textual form to reduce disturbance. 

%Other studies have shown that it is possible to enhance community content contribution by providing subtle interfaces without disruptive character or radical representation changes~\cite{Wikipedia}. These interfaces stimulated the secondary task of editing Wikipedia articles rather than passive consumption as the primary task. Enhancing user engagement is also the topic of 

It is important to consider the distinction between engagement and such disturbance. When applied to designing user interfaces, active pop-up comments have been found to be highly useful to designers, but also somewhat disruptive to their concentration~\cite{CommentingSystems}. An interface that has too many moving elements may disturb the reader's focus more than engage them or supplement the material. However, some degree of disturbance is necessary for visibility. the active decision of commenting is a secondary task to the user's main goal of passively reading the article. Users are more likely to engage in a secondary task if the threshold of engaging with that task is low. However, methods that lower barriers to the task most effectively, like pop-ups, are also highly disruptive~\cite{Wikipedia}. Improving comments must thus optimize for minimal disruption and maximal engagement.
Other studies have looked into ways to complement commenting systems. \textit{Reflect} is a platform that enables users to summarize each others comments in order to enhance comprehension and provide feedback on comments~\cite{Reflect}. \textit{Balancer} is a complementary widget that provides information on how politically skewed a user's aggregate source of news is~\cite{Politics}. These studies reveal that there exists a need in current commenting platforms to enhance engagement so that users better comprehend the reference media and consider the views of others.
