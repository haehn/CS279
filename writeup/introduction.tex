\section{Introduction}

\subsection{Motivation}

Many websites support the ability to give content-related feedback in the form of comments. Such websites include news pages, media sites, online shops, blogs and social networks. Both reading and adding comments can engage the user in information exchange, personal reflection and lively discussion. In addition, such engagement can impact real world actions like voting, purchasing a product, making choices or participating in a cause.

Traditional commenting systems are featured at the bottom of the content. This vertical structure impedes the reader's ability to view comments alongside the reference media. These systems also lack a method of referencing sections of the original media. Furthermore, the popular filter mechanism of up-voting popular comments to enhance their visibility risks suppressing newer, and inevitably less popular, comments.

Previous research on annotation and commenting systems has focused primarily on one of three goals: enhancing reader comprehension~\cite{NewsInterfaces}~\cite{NB}~\cite{Reflect}, increasing critical analysis and reflection~\cite{AnnotationsStudents}, and encouraging a diversity of viewpoints~\cite{OpinionSpace}~\cite{Politics}. Separately, the ends of these goals comprise critical components of reader engagement. The value of this engagement can be seen from how these goals benefit the reader. 
Enhanced comprehension provides a clearer and more direct stream of knowledge to the reader and adds value to the content of the article. 
Critical analysis and reflection encourage readers to construct and develop their own opinions from the reading. 
With access to a broader spectrum of perspectives, readers are urged to consider alternate opinions and as a result, to adapt, strengthen, or clarify their own. 

In our study, we measure and take into consideration overall engagement. When engagement as a whole is optimized, readers will have a clearer understanding of the article, develop their own viewpoints as they step back from the article, challenge these viewpoints as they come across other opinions, and feel motivated to take action outside of the scope of the article. Ultimately, greater engagement encourages readers to step away from the article, taking what they have learned with a clearer perspective and engaging more knowledgeably and more openly with the outside world.

\subsection{Approach}

We explore how different commenting models encourage and facilitate engagement by proposing a new system. As opposed to traditional systems, our system \textbf{MindMargin} displays comments in a horizontal infinite scroll alongside the reference media. We compare the two systems by quantifying the quality of readers' comments and their follow-up actions. We take into account the ratio of substantial responses to overall responses, instances of referencing the text correctly/incorrectly, and follow-up actions.

We have several hypotheses regarding the comparison of these two interfaces:
\begin{enumerate}
\item Readers will generate more substantial comments using MindMargin
\item MindMargin reduces instances of misquoting and confusion by referring directly to the reference media
\item MindMargin increases user engagement with the reference media and motivates follow-up action
\end{enumerate}

Hypothesis 1 is motivated by~\cite{AnnotationsStudents} that found that students engaged more critically with the text when the text was accompanied by annotations. 
Hypothesis 2 is motivated by~\cite{NB}, in which the comprehension of students reading a text increased with the geographic placement of annotations next to the text. This horizontal structure prevented comments that were out of context and drew students' attention to relevant comments. The authors also found that this increased comprehension encouraged student response and participation.
Hypothesis 3 combines the findings of the two previous hypotheses and considers how the increase and optimization of overall engagement induces action and thought beyond the original intended scope of the article. 

\subsection{Contributions}

This project includes the following novel contributions:

\begin{itemize}
\item a horizontally structured user interface for anchored comments on websites
\item a metric to ensure newer comments are still visible beside popular comments while non-constructive comments are less visible
\item insights how comments can enhance user engagement regarding
\begin{enumerate}
\item online discussions
\item resulting actions (ideally in offline scenarios)
\end{enumerate}

\end{itemize}