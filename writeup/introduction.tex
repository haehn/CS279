\section{Introduction}

\subsection{Motivation}

Many websites support the ability to give content-related feedback in the form of comments. Such websites include news pages, media sites, online shops, blogs and social networks. Both reading and adding comments can engage
the user in information exchange, personal reflection and lively discussion. In addition, such engagement can impact real world actions like voting, purchasing a product, making choices or participating in a cause. 

Traditional commenting systems are featured at the bottom of the content which impedes the reader's ability to view comments alongside the reference media. They also lack a mechanism to reference sections of the original media. Furthermore, it is common to up-vote popular comments to enhance their visibility but this risks suppressing newer, but less popular, comments.

\subsection{Approach}

We explore how different commenting models encourage and facilitate engagement by proposing a new system. As opposed to traditional systems, our system \textbf{MindMargin} displays comments in a horizontal infinite scroll alongside the reference media. We compare the two systems by quantifying the quality of readers' comments and their follow-up actions. We take into account the ratio of substantial responses to overall responses, instances of referencing the text correctly/incorrectly, and follow-up actions.

We have several hypotheses regarding the comparison of these two interfaces:
\begin{enumerate}
\item Readers will generate more substantial comments using MindMargin
\item MindMargin reduces instances of misquoting and confusion by referring directly to the reference media
\item MindMargin increases user engagement with the reference media and motivates follow-up action
\end{enumerate}

\subsection{Contributions}

This project includes the following novel contributions:

\begin{itemize}
\item a horizontally structured user interface for anchored comments on websites
\item a metric to ensure newer comments are still visible beside popular comments while non-constructive comments are less visible
\item insights how comments can enhance user engagement regarding
\begin{enumerate}
\item online discussions
\item resulting actions (ideally in offline scenarios)
\end{enumerate}

\end{itemize}