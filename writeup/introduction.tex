\section{Introduction}

\subsection{Motivation}

Many websites support the ability to give content-related feedback in forms of comments. Such websites include news pages, media sites, online shops, blogs and social networks. Both reading and adding comments can engage
the user in information exchange, personal reflections and lively discussions. In addition, such engagement can impact real world actions like voting, purchasing a product, making choices or participating in a cause. 

Traditional commenting systems are featured at the bottom of the content which impedes the ability of viewing comments and the reference media in parallel. They also lack a mechanism to reference sections of the original media. Further, it is common to up-vote popular comments which then appear higher than others with the risk of suppressing newer but less popular comments.

\subsection{Approach}

We explore how differing commenting models encourage and facilitate engagement by measuring the quality of the comments generated and the actions that the readers were motivated to take from engaging with the reading. We have proposed a new interface for comments on web-based articles that displays comments in a horizontal infinite scroll alongside the text, as opposed to the typical vertical commenting interface appearing beneath the text.
 
We have several hypotheses concerning the comparison of these two interfaces: 1) readers will generate more substantial comments in the horizontal interface, 2) readers will in the horizontal interface generate more comments that refer directly to a part of the text, thus reducing instances of misquoting and confusion, 3) engagement in the horizontal interface would be powerful enough to incite more people to change their point of view and to apply their knowledge from the article by taking action in their real lives.
 
This comparison would reveal the impact that deep engagement in a reading can have on individuals' motivations and perceptions of the world, and how a new commenting platform can augment that engagement.
 
We plan to use sharing an article as a measurement of engagement. Other measurements of engagement we hope to explore in the future are a reader's motivation a) to vote as a result of reading politically charged articles, b) to donate or volunteer his/her time or resources from reading an article about a disease or charity, and c) to be urged to buy or to avoid buying a product because of a compelling product review.


\subsection{Contributions}

This project includes the following novel contributions:

\begin{itemize}
\item a horizontally structured user interface for commenting on websites with anchoring features
\item a metric to ensure newer comments are still visible beside popular comments while non-constructive comments are less visible
\item insights how comments can enhance user engagement regarding
\begin{enumerate}
\item online discussions
\item resulting actions (ideally in offline scenarios)
\end{enumerate}

\end{itemize}